\section{Five families}
\begin{questions}
	\question If a group is generated by just one elemnt, what kind of group is it?
	\begin{solution}
		A cyclic group
	\end{solution}

	\question
	\begin{parts}
		\part In the group $C_5$, compute $2+2$.
		\begin{solution}
			\par $2+2 = 4$
		\end{solution}

		\part In the group $C_5$, compute $4+3$.
		\begin{solution}
			\par $4+3 = 2$
		\end{solution}
		
		\part In the group $C_{10}$, compute $8+7$.
		\begin{solution}
			\par $8+7 =5$
		\end{solution}
		
		\part In the group $C_{10}$, compute $9+1$.
		\begin{solution}
			\par $9+1 = 0$
		\end{solution}
		
		\part In the group $C_3$, compute $2+2+2+2+2+2$.
		\begin{solution}
			\par $2+2+2+2+2+2 = 0$
		\end{solution}
		
		\part In the group $C_{11}$, compute $10-8+1-7+6+5$.
		\begin{solution}
			\par $10-8+1-7+6+5 = -4 = 7$
		\end{solution}
	\end{parts}

	\question For each statement below, determine if it is true or false.
	\begin{parts}
		\part Every cyclic group is abelian.
		\begin{solution}
			\par True.
		\end{solution}

		\part Ever abelian group is cyclic.
		\begin{solution}
			\par False, consider $V_4$
		\end{solution}
		
		\part Every dihedral group is abelian.
		\begin{solution}
			\par False, consider $D_4$, then $\sigma \circ \tau_X \not = \tau_X \circ \sigma$.
		\end{solution}
		
		\part Some cyclic groups are dihedral.
		\begin{solution}
			\par False, it would imply that some dihedral groups are abelian. Which is false.
		\end{solution}
		
		\part There is a cyclic group of order 100.
		\begin{solution}
			\par True, $C_{100}$.
		\end{solution}

		\part There is a symmetric group of order 100.
		\begin{solution}
			\par False, $|S_4| = 4! = 24$ while $|S_5| = 5! = 120$
		\end{solution}
		
		\part If some pair of elements in a group commute, the group is abelian.
		\begin{solution}
			\par False, consider $D_4$.
		\end{solution}
		
		\part If every pair of elements in a group commute, the group is cyclic.
		\begin{solution}
			\par False, consider $V_4$.
		\end{solution}
		
		\part If the pattern on the left of Figure 5.8 appears nowhere in the Cayley diagram for a group, then the group is abelian.
		\begin{solution}
			\par True.
		\end{solution}
	\end{parts}
	
	\question
	\begin{parts}
		\part Use the Cayley diagram of the group $D_5$ in Figure 5.17 to compute $r\cdot f\cdot r$ in that group.
		\begin{solution}
			$r\cdot f \cdot r =f$
		\end{solution}

		\part Is the answer the same or different if you do the computation in the group $D_3$ instead?
		\begin{solution}
			$r\cdot f \cdot r =f$
		\end{solution}
		
		\part Is the answer the same or different if you do the computation in the group $D_n$ instead?
		\begin{solution}
			\par $r\cdot f \cdot r = f$, because $r\cdot f = r^{-1} \cdot ? = f\cdot r \cdot f \cdot ?$, which results in $? = f^{-1}=f$.
		\end{solution}
	\end{parts}

	\question Compare the strengths and weaknesses of the tree visualization techniques introduced in this book: Cayley diagrams, multiplication tables, and cycle graphs.
	
	\question Sketch the following visualizations.
	\begin{parts}
		\part a cycle graph for $C_9$
		\part a Cayley diagram for $D_4$
		\part a multiplication table for $D_2$
		\begin{solution}
			\[
				\begin{array}{|c|c|}
					\hline
					0 & 1 \\
					\hline
					1 & 0 \\
					\hline
				\end{array}
			\]
		\end{solution}
	\end{parts}

	\question Describe in words what each of the following visualizations look like for $C_{999}$
	\begin{parts}
		\part Cayley diagram
		\part multiplication table
		\part cycle graph
	\end{parts}

	\question Describe in words what each of the following visualizations look like for $D_{999}$
	\begin{parts}
		\part Cayley diagram
		\part multiplication table
		\part cycle graph
	\end{parts}
	
	\question What are the orders for the first ten symmetric groups, $S_1$ through $S_{10}$? What are the orders of their corresponding alternating groups, $A_1$ trough $A_{10}$? Explain your answer for the order of $A_1$.
	\begin{solution}
		\par $|S_n| = n!$
		
		\par Since $A_1$ is constructed by considering the squares of elements from $S_1$ one counts $|A_1| = 1$.
		
		\par $|A_2| = 1$ while $|S_2| = 2$,
		\par $|A_3| = 3$ (Figure 5.25) while $|S_3| = 3! = 6$
		\par $|A_4| = 12$
		\par $|A_n| = n! : 2$
	\end{solution}

	\question The exercises for Chapter 3 asked you to create several Cayley diagrams. This chapter introduced a mehtod for telling whether a group is abelian based on its Cayley diagram. For each of the Chapter 3 exercises mentioned below, first determine whether the group belongs to any of the five families introduced in this chapter, and if so, what the group's name is (e.g. $D_4, S_3,$ etc.). Explain how you determine each of your answers.


	\question Explain why every cyclic group is abelian.
	\begin{solution}
		\par Rotations around a fixed center is a commutative operation.
	\end{solution}
	
	\question Qhy is it sufficient, when looking to see if a Cayley diagram represents an abelian group, to only consider the arrows? Why do we not need to examine every possible combination of paths?
	\begin{solution}
		\par Because the path is a constructed as a sequence of arrows.
	\end{solution}
	
	\question 
	\begin{parts}
		\part Create a cycle graph for the group $V_4$ using the multiplication table in Figure 5.31.
		\begin{solution}
			petal flower with three petals
		\end{solution}

		\part Create a cycle graph for the group $A_4$ using the Cayley diagram in Exercise 4.6 part (c).
		\begin{solution}
			Four 2 node petals, 3 1 node petal
		\end{solution}
	\end{parts}

	\question
	\begin{parts}
		\part Is there a dihedral group of order 7?
		\begin{solution}
			\par No $|D_n| = 2n$
		\end{solution}

		\part If $A_n$ has order $2520$, what is $n$?
		\begin{solution}
			\par $7, 7! = 5040$
		\end{solution}

		\part If $A_n$ has order $m$, what order does $S_n$ have?
		\begin{solution}
			$2m$
		\end{solution}
	\end{parts}

	\question For each part below, compute the orbit of the element in the group. Your answer will be alist of elements from the group that ends with the identity.
	\begin{parts}
		\part The element $r^2$ in the group $D_{10}$
		\begin{solution}
			\par $= \{ r^2, r^4, r^6, r^8, e\}$
		\end{solution}

		\part The element $10$ in the group $C_{16}$
		\begin{solution}
			\par $= \{ 10, 4, 14, 8, 2, 12, 6, 0\}$
		\end{solution}

		\part The element $25$ in the group $C_{30}$
		\begin{solution}
			\par $= \{25, 20, 15, 10, 5, 0 \}$
		\end{solution}

		\part The element $12$ in the group $C_{42}$
		\begin{solution}
			\par $=\{12, 24, 36, 6, 18, 30, 0 \}$
		\end{solution}

		\part The element $s$ in the group whose Cayley diagram is on the left below. (Assume the element $a$ at the top left is the identity.)
		\begin{solution}
			\par $=\{ s, m, j, a\}$
		\end{solution}

		\part The element $l$ in the group whose Cayley diagram is on the right below. (Assume the element $a$ at the top is the identity.)
		\begin{solution}
			\par $=\{ l, e, p, a\}$
		\end{solution}
	\end{parts}

	\question Recall the notation for generators from Exercise 4.25. Use it to fill in the blanks below with however many elements necessary to generate the group. Use as few elements as possible.
	\begin{parts}
		\part $C_n= \{0,1,\ldots, n-1\}$
		\begin{solution}
			\par $= \langle 1\rangle$
		\end{solution}
		
		\part $D_n=\{ e,r,\ldots r^{n-1}, f,fr,\ldots, fr^{n-1}\}$
		\begin{solution}
			\par $= \langle r,f\rangle$
		\end{solution}
	\end{parts}
	
	\question Create multiplication tables for the smallest dihedral groups $D_1, D_2, D_3$ and so on, until you find the first non-abelian member of the family. Which is it an how can you tell?
	\begin{solution}
		\par $D_1$
		\[
			\begin{array}{|c|}
			\hline
			e\\
			\hline
			\end{array}
		\]
		\par $D_2$
		\[
			\begin{array}{|c|c|}
			\hline
			e& r\\
			\hline
			r & e\\
			\hline
			\end{array}
		\]
		\par $D_3$
		\[
			\begin{array}{|c|c|c|c|c|c|}
			\hline
			e & r & r^2 & f & rf & r^2f\\
			\hline
			r & r^2 & e & rf & r^2 f & f\\
			\hline
			r^2 & e & r & r^2f & f & rf\\
			\hline
			f & r^2f & rf & e & r^2 & r\\
			\hline
			rf & f & r^2f & r & e & r^2\\
			\hline
			r^2f & rf & f & r^2 & r & e\\
			\hline 
			\end{array}
		\]
		\par Is niet Abels. Zo geldt bijvoorbeeld dat $fr = r^2f$ terwijl $rf = rf$.
	\end{solution}
	
	\question Repeat Exercise 5.17 for the symmetric groups $S_n$. Use the permutation notation from this chapter.
	\begin{solution}
		\par $S_1$
		\[
			\begin{array}{|c|}
			\hline
			e\\
			\hline
			\end{array}
		\]
		\par $S_2$
		\[
			\begin{array}{|c|c|}
			\hline
			e& (1 2)\\
			\hline
			(1 2) & e\\
			\hline
			\end{array}
		\]
		\par $S_3$
		\[
			\begin{array}{|c|c|c|c|c|c|}
			\hline
			e& (1 2)(3) & (1)(2 3) & (1 3)(2) & (1 2 3) & (1 3 2)\\
			\hline
			(1 2)(3) & e & (123) & (132) & (1)(23)&(13)(2) \\
			\hline
			(1)(23) & (132) & e & (123) & (13)(2) & (12)(3)\\
			\hline
			(13)(2) & (123) & (132) & e & (12)(3) & (1)(23)\\
			\hline
			(123) & (13)(2) & (12)(3) & (1)(23) & (132) & e\\
			\hline
			(132) & (1)(23) & (13)(2) & (12)(3) & e & (123)\\
			\hline
			\end{array}
		\]
		\par Is niet Abels.
	\end{solution}

	\question For each symmetric group whose multiplication table you created in Exercise 5.18, compute the elements of the corresponding alternating group, as in Figure 5.25. For each alternating group you compute, create
	\begin{parts}
		\part a multiplication table,
		\begin{solution}
			\par $A_1 = S_1$
			\par $A_2 = S_1$
			\par $A_3 = \{ e, (123), (132)\}$
			\[
				\begin{array}{|c|c|c|}
				\hline
				e& (1 2 3) & (132)\\
				\hline
				(123) & (132) & e\\
				\hline
				(132) & e & (123)\\
				\hline
				\end{array}
			\]
		\end{solution}

		\part a Cayley diagram, and
		\part a cycle graph
	\end{parts}

	\question Some of the smallest meembers of the families $C_n,D_n, S_n$ and $A_n$ actually belong to more than one family, as long as we do not care about the names of the elements, but about the group structure. For instance, $D_1$ is a group with two elements, and its mulitplicationt able has the same pattern as that of $C_2$, as shown here.
	\par What other groups belong to more than one of the families we studied in this chapter? (Another way to read this question is, ``are there any groups of the families $C_n, D_n, S_n,$ or $A_n$ that are isomorphic to a group in another of those families?'')
	
	\begin{solution}
		\par $D_1\cong S_1\cong C_1 \cong A_1$
		\par $D_2\cong S_2 \cong C_2$
		\par $C_3\cong A_3$
		\par $D_3\cong S_3$
	\end{solution}

	\question For each of the following questions, either exhibit a group that answers the question in the affirmative or give a clear explanation of thy the answer to the question is negative.
	\begin{parts}
		\part Is here a cyclic group with exactly four generators? (Not that it takes four elements to generate the group, but that there are four different elements $a,b,c,d$ in $C_n$ and $C_n = \langle a\rangle =\langle b\rangle =\langle c\rangle \langle d\rangle.$) Is there more than one such group?
		\begin{solution}
			\par Yes, $C_8$ with generators $\{ 1,3,5,7\}$
			\par $C_5$ with generators $\{1,2,3,4 \}$ complies as well.
		\end{solution}
	
		\part Is there a cyclic group with exactly one generator? Is there more than one?
		\begin{solution}
			\par Both $C_1 = \langle 0\rangle$ and $C_2=\langle 1\rangle$ comply.
		\end{solution}
	\end{parts}
	
	\question Group Explorer
	
	\question Broad
	
	\question This chapter gave regular polygons as examples of objects whose symmetries are described by dihedral groups, that is, objects with both rotational and bilateral symmetry, but no other symmetries. What other objects fit in this category?
	\begin{solution}
		\par Snowflakes, rounded polygons, \ldots
	\end{solution}
	
	\question Analyze the symmetries of a tetrahedro using the technique from Defintion 3.1, resulting in the Cayley diagram for its symmetry group. Here are a few hints to get you started.
	
	\question As you now from this chapter, $S_3$ and $D_3$ are two different names for the same group. Yet no larger dihedral group is also a symmetric group. Give an argument based on the physical featers of a $n$-gon for why this is so? $(n\geqslant 4)$.
	\begin{solution}
		\par $n! \geqslant 2n$ when $n\geqslant 4$. A regular $n$-gon has exactly $n$ axis of symmetry, and $n$ rotations resulting in $2n$ transformations.
	\end{solution}
	
	\question Section 5.2.3 describes what the cycle graph will look like for $C_p\times C_p$ if $p$ is a prime number. Draw the cycle graph for $C_5\times C_5$. (It is not necessary to label the elements.)
	\begin{solution}
		\par d
	\end{solution}


\end{questions}